\documentclass{article}
\usepackage{pdflscape}
\usepackage{afterpage}

\usepackage{capt-of}

\usepackage{tabularray}
\usepackage{amsmath}
\usepackage{bm}
\usepackage{geometry}
\usepackage{xcolor}


\renewcommand{\rmdefault}{\sfdefault}
%\usepackage{times}

\newcommand{\wm}[2]{%
	\begin{minipage}{#1\textwidth}
		\centering
        #2
	\end{minipage}%
}

\def\yels{yellow!80!white}
\def\blus{blue!30!white}

\usepackage{lipsum}% dummy text

\begin{document}


\afterpage{%
	\clearpage
	\thispagestyle{empty}
	\newgeometry{left=6mm, right=1mm, bottom=10mm, top=10mm}
	\begin{landscape}
		\centering
		\begin{tblr}{
				colspec={
					|[0.2em]X[c,0.5]|[0.2em]X[c,1]|[0.2em]
					X[c,2]X[c,2]X[c,2]|[0.2em]
					X[c,2]X[c,2]X[c,2]|[0.2em]
					X[c,2]X[c,2]X[c,2]|[0.2em]
					X[c,2]X[c,2]|[0.2em]
					X[c,2]X[c,2]X[c,2]|[0.2em]},
				hlines, vlines,
				rows={ht=1\baselineskip},
				row{1} = {1.7\baselineskip,font=\bfseries, c, m},
				row{2} = {3\baselineskip, c, m},
				cell{4-Z}{4,5,7,8,10,11,13,15,16} = {fg=black!50!white},
				hline{1,Z,3,4,2} = {0.2em},
				cells={valign=m, halign=c}
			}
			\SetCell[c=16]{c} {P\kern0.2em O\kern0.2em S\kern0.2em I\kern0.2em  T\kern0.2em  I\kern0.2em  V\kern0.2em  E}\\
			\SetCell[r=2]{m} \# & & \SetCell[c=3]{c} \wm{0.15}{Pressure Calibrator\\[2pt] \textcolor{red}{{\textbf{MAX 75 kPa}}}} & & &   \SetCell[c=3]{c} Bourdon Gauge 1 & & & \SetCell[c=3]{c} Bourdon Gauge 2 & & & \SetCell[c=2]{c} \wm{0.15}{Bundenberg\\[1pt] Pressure Gauge} & & \SetCell[c=3]{c} \wm{0.2}{Hg Glass Manometer (+) \\[2pt] \textcolor{red}{{\textbf{MAX 32 cm Hg}}}}  \\
			&  & \SetCell{bg=\yels,fg=black}\textbf{kPa} & bar & bar $P_{abs}$ &  \SetCell{bg=\yels,fg=black} \textbf{psi} & bar & bar $P_{abs}$  & \SetCell{bg=\yels,fg=black} kN/m$^2$ & \textbf{bar} & bar $P_{abs}$  &\SetCell{bg=\yels,fg=black} bar & bar $P_{abs}$ & \SetCell{bg=\yels,fg=black}\textbf{cm Hg} & bar & bar $P_{abs}$ \\
			1  &  & 0 & 0 & 1.013 & 1 & 0.069 & 1.082 & 1 & 0.01 & 1.023 & -0.05 & 0.963 & 0.4 & 0.005 & 1.018 \\
			2  &  & 5.7 & 0.057 & 1.070 & 2 & 0.138 & 1.151 & 8 & 0.08 & 1.093 & 0 & 1.013 &  3.5 & 0.047 & 1.060 \\
			3  &  & 10.4 & 0.104 & 1.117 & 2.6 & 0.179 & 1.192 & 14 & 0.14 & 1.153 & 0.04 & 1.053 & 5.3 & 0.071 & 1.084 \\
			4  &  & 16 & 0.160 & 1.173 & 3.4 & 0.234 & 1.247 & 20 & 0.20 & 1.213 & 0.10 & 1.113 & 7.4 & 0.099 & 1.112 \\
			5  &  & 21.1 & 0.211 & 1.224 & 4.1 & 0.283 & 1.296 & 25 & 0.25 & 1.263 & 0.15 & 1.163 & 9.4 & 0.125 & 1.138 \\
			6  &  & 27.7 & 0.277 & 1.290 & 5 & 0.345 & 1.358 & 30 & 0.30 & 1.313 & 0.22 & 1.233 & 11.6 & 0.155 & 1.168 \\
			7  &  & 34.2 & 0.342 & 1.355 & 6 & 0.414 & 1.427 & 39 & 0.39 & 1.403 & 0.29 & 1.303 & 14.2 & 0.189 & 1.202 \\
			8  &  & 40 & 0.400 & 1.413 & 6.8 & 0.469 & 1.482 & 45 & 0.45 & 1.463 & 0.35 & 1.363 & 16.4 & 0.219 & 1.232 \\
			9  &  & 46.1 & 0.461 & 1.474 & 7.6 & 0.524 & 1.537 & 50  & 0.50 & 1.513 & 0.4 & 1.413 & 18.7 & 0.249 & 1.262 \\
			10 &  & 52.2 & 0.522 & 1.535 & \SetCell{bg=\blus,fg=black} 8.5 & 0.586 & 1.599 & \SetCell{bg=\blus,fg=black} 57  & 0.57 & 1.583 &\SetCell{bg=\blus,fg=black} 0.47 & 1.483 &21 & 0.280 & 1.293 \\
		\end{tblr}
		
		\vspace*{1em}
		
		\begin{tblr}{
				colspec={
					|[0.2em]X[c,0.5]|[0.2em]X[c,1]|[0.2em]
					X[c,2]X[c,2]X[c,2]|[0.2em]
					X[c,2]X[c,2]X[c,2]|[0.2em]
					X[c,2]X[c,2]X[c,2]|[0.2em]
					X[c,2]X[c,2]|[0.2em]
					X[c,2]X[c,2]X[c,2]|[0.2em]},
				hlines, vlines,
				rows={ht=1\baselineskip},
				row{1} = {1.7\baselineskip,font=\bfseries, c, m},
				row{2} = {3\baselineskip, c, m},
				hline{1,Z,3,4,2} = {0.2em},
				cell{4-Z}{4,5,7,8,10,11,13,15,16} = {fg=black!50!white},
				cells={valign=m, halign=c}
			}
			\SetCell[c=16]{c} {N\kern0.2em E\kern0.2em G\kern0.2em A\kern0.2em  T\kern0.2em  I\kern0.2em  V\kern0.2em  E}\\
			\SetCell[r=2]{m} \# &\SetCell[r=2]{m} \rotatebox{90}{\wm{0.1}{Reference\\ Value}} & \SetCell[c=3]{c} \wm{0.15}{Pressure Calibrator\\[2pt] \textcolor{red}{{\textbf{MAX 75 kPa}}}} & & &   \SetCell[c=3]{c} Bourdon Gauge 1 & & & \SetCell[c=3]{c} Bourdon Gauge 2 & & & \SetCell[c=2]{c} \wm{0.15}{Bundenberg\\[1pt] Pressure Gauge} & & \SetCell[c=3]{c} \wm{0.2}{Hg Glass Manometer (+) \\[2pt] \textcolor{red}{{\textbf{MAX 32 cm Hg}}}}  \\
			&  & \SetCell{bg=\yels,fg=black}\textbf{kPa} & bar & bar $P_{abs}$ &  \SetCell{bg=\yels,fg=black} \textbf{psi} & bar & bar $P_{abs}$  & \SetCell{bg=\yels,fg=black} kN/m$^2$ & \textbf{bar} & bar $P_{abs}$  &\SetCell{bg=\yels,fg=black} bar & bar $P_{abs}$ & \SetCell{bg=\yels,fg=black}\textbf{cm Hg} & bar & bar $P_{abs}$ \\
			1  &  & 0 & 0 & 1.013 & 1.2 & 0.083 & 1.096 & 2.5 & 0.025 & 1.038 & -0.05 & 0.963 & 0.4 & 0.005 & 1.018 \\
			2  &  & -5.6 & -0.056 & 0.957 & 0.4 & 0.028 & 1.041 & -1 & -0.01 & 1.003 & -0.1 & 0.913 &  -0.7 & -0.009 & 1.004 \\
			3  &  & -12.1 & -0.121 & 0.892 & -0.5 & -0.034 & 0.979 & -9 & -0.09 & 0.923 & -0.16 & 0.853 & -3.7 & -0.049 & 0.964 \\
			4  &  & -18 & -0.180 & 0.833 & -2 & -0.138 & 0.875 & -15 & -0.15 & 0.863 & -0.24 & 0.773 & -5.4 & -0.072 & 0.941 \\
			5  &  & -21.8 & -0.218 & 0.795 & -2.8 & -0.193 & 0.820 & -20 & -0.20 & 0.813 & -0.27 & 0.743 & -6.8 & -0.091 & 0.922 \\
			6  &  & -25.4 & -0.254 & 0.759 & -4 & -0.276 & 0.737 & -23 & -0.23 & 0.783 & -0.3 & 0.713 & -8.2 & -0.109 & 0.904 \\
			7  &  & -29.3 & -0.293 & 0.720 & -6 & -0.414 & 0.599 & -27 & -0.27 & 0.743 & -0.35 & 0.663 & -9.6 & -0.128 & 0.885 \\
			8  &  & -33.6 & -0.336 & 0.677 & -7.1 & -0.490 & 0.523 & -32 & -0.32 & 0.693 & -0.4 & 0.613 & -11.3 & -0.151 & 0.862 \\
			9  &  & -37.6 & -0.376 & 0.637 & -8.3 & -0.572 & 0.441 & -36 & -0.36 & 0.653 & -0.44 & 0.573 & -12.8 & -0.171 & 0.842 \\
			10 &  & -41.7 & -0.417 & 0.596 & \SetCell{bg=\blus,fg=black} -9.5 & -0.655 & 0.358 & \SetCell{bg=\blus,fg=black} -40  & -0.40 & 0.613 &\SetCell{bg=\blus,fg=black} -0.49 & 0.523 &-14.4 & -0.192 & 0.821 \\
		\end{tblr}
		
		\raggedright
		
		\vspace*{1em}
		
		Mercury manometer Pressure = Density $\times$ Gravity $\times$ Height in metres = $13600\times9.81\times$Hg height in metres\\[2pt]
		Atmospheric Pressure (Patm) from the Digital Manometer This is a fill-in line with a specific length: \underline{\phantom{ssssssssssssss}}.mbar (1000 mbar = 1 bar)\\[2pt]
		\colorbox{\blus}{\textcolor{black}{Bordon Gauge 1, Bourden Gauge 2, and Bundenberg Gauge offsets}}
		
	\end{landscape}
	\clearpage
}
\restoregeometry

\newpage

\section*{Explanation of Calculations}

\begin{enumerate}
	\item For the bar columns, I converted from the base unit in each section:
	\begin{itemize}
		\item kPa $\Rightarrow$ bar: Divided by 100, i.e., \(\text{Pressure}_{\text{bar}} = \frac{\text{Pressure}_{\text{kPa}}}{100}\)
		\item psi $\Rightarrow$ bar: Multiplied by 0.06895, i.e., \(\text{Pressure}_{\text{bar}} = \text{Pressure}_{\text{psi}} \times 0.06895\)
		\item kN/m$^2$ $\Rightarrow$ bar: Divided by 100 (since \(1 \, \text{kN/m}^2 = 1 \, \text{kPa}\)), i.e., \(\text{Pressure}_{\text{bar}} = \frac{\text{Pressure}_{\text{kN/m}^2}}{100}\)
		\item cm Hg $\Rightarrow$ bar: Multiplied by 0.01333, i.e., \(\text{Pressure}_{\text{bar}} = \text{Pressure}_{\text{cm Hg}} \times 0.01333\)
	\end{itemize}
	
	\item For the ``bar $P_{abs}$'' columns, I added the standard atmospheric pressure of \(1.013 \, \text{bar}\) to the gauge pressure values:
	\[
	\text{P}_{\text{abs}} = \text{P}_{\text{gauge}} + 1.013 \, \text{bar}
	\]
	
	\item All values are formatted to 3 decimal places for consistency.
	
	\item I preserved all the highlighted cells (with the blue and yellow backgrounds) exactly as in your original table.
\end{enumerate}

The table is now complete and ready to be included in document.


\end{document}
