\documentclass{article}
\usepackage{graphicx} 
\usepackage{float}
\usepackage{booktabs}
\usepackage{array}
\usepackage{arydshln}
\usepackage{siunitx}
\usepackage{hyperref}
\usepackage{cancel}
\usepackage{changepage}
\usepackage{placeins}
\usepackage{enumitem}
\usepackage{siunitx}
%\usepackage{showframe}
\usepackage{times}

\usepackage{tabularx}
\usepackage{amsmath, amssymb, amscd, MnSymbol, mathrsfs}
\usepackage{cellspace}
\usepackage{tikz}
\usetikzlibrary{calc, patterns, angles, quotes, decorations.markings, decorations.pathmorphing, hobby}
\usepackage{xfrac}

\usepackage{chemfig}
\usepackage{caption}
\usepackage{tcolorbox}
\usepackage{bm}
\usepackage{pdfpages}
\usepackage{empheq}
\usepackage{pgfplots}
\pgfplotsset{compat=1.18}
\usepackage[oldvoltagedirection]{circuitikz}
\usepackage{microtype}
\usepackage{tikz-3dplot}
\usepackage{textcomp}
% Custom commands
\newcommand{\vect}[1]{\boldsymbol{\mathbf{#1}}}
\newcolumntype{C}{>{\centering\arraybackslash}X}
\newcolumntype{M}[1]{>{\centering\arraybackslash}m{#1}}

\usetikzlibrary{external}
\tikzexternalize[prefix=figures/]

\newcommand\myfrac[2]{\sfrac{#1\mkern-1.2mu}{#2}}
\usepackage{xcolor}

% Define custom colors
\definecolor{darkblue}{rgb}{0.1,0.1,0.5} % A dark blue shade
\definecolor{formalshade}{rgb}{0.95,0.95,1} % A light blue shade for the background

% For the adjustwidth environment
\PassOptionsToPackage{strict}{changepage}
\usepackage{changepage}

% For formal definitions
\usepackage{framed}

\newcommand{\formalsource}{} % Initialize an empty macro to store the source text

\newenvironment{formal}[3][]{% Start of the environment
	\renewcommand{\formalsource}{#1}% Store the optional argument
	\def\FrameCommand{%
		\hspace{1pt}%
		{\color{#2}\vrule width 2pt}%
		{\color{#3}\vrule width 4pt}%
		\colorbox{#3}%
	}%
	\MakeFramed{\advance\hsize-\width\FrameRestore}%
	\noindent\hspace{-4.55pt}% Disable indenting the first paragraph
	\begin{adjustwidth}{}{7pt}%
		\vspace{2pt}%
	}%
	{%
		\vspace{4pt}%
		\ifx\formalsource\empty % Check if the source is empty
		\else
		\hfill{\footnotesize{\formalsource}}% Align source to the bottom-right
		\fi
	\end{adjustwidth}\endMakeFramed%
}


% Custom itemize list with images for positive and negative items
\newlist{gitemize}{itemize}{1} % Just one level for the list
\setlist[gitemize,1]{
	leftmargin=2.8em, % Adjust the margin for the list
	labelsep=1em % Control the space between the label and the list item
}

% Define checkmark and cross symbols for positive and negative items
\newcommand{\checkitem}{\raisebox{-0.25\height}{\includegraphics[width=0.4cm]{checkmark.png}}}
\newcommand{\crossitem}{\raisebox{-0.25\height}{\includegraphics[width=0.4cm]{cross.png}}}


\usepackage[left=0.8in,right=0.8in,top=0.5in,bottom=0.69in,includeheadfoot,letterpaper]{geometry}
\usepackage{fancyhdr}
\usepackage{graphicx}
\usepackage{tabularray}
\usepackage{varwidth} 


\newcommand{\wm}[2]{%
	\begin{minipage}{#1\textwidth}
		\centering
		#2
	\end{minipage}%
}

\pagestyle{fancy}
\fancyhf{}


\renewcommand{\headrulewidth}{0.4pt}
\renewcommand{\footrulewidth}{0.4pt}

\fancyhead[L]{\includegraphics[height=1.2cm]{images/Kingston_University_London_logo_200-tablet.png}}
\fancyhead[R]{EG4024 – ME – Fluid Mechanics and Thermodynamics}
\fancyfoot[C]{Department of Mechanical Engineering}
\fancyfoot[R]{\thepage}

\usepackage{scalerel}

\setlength{\headheight}{30pt}
\setlength{\footskip}{20pt}



\usepackage[export]{adjustbox}
\usepackage{tocloft}
\renewcommand{\cfttoctitlefont}{}
\renewcommand{\contentsname}{}
\renewcommand{\cftsecleader}{\cftdotfill{\cftdotsep}}

\setlength{\cftbeforesecskip}{0.5em}


\usepackage{xcolor}      % For color options
\usepackage{hyperref}    % For hyperlinks
\usepackage{xurl}        % For better URL handling
\hypersetup{
	colorlinks=true,
	linkcolor=blue!50!black,
	urlcolor=blue,       % Color for URLs
}



%Refer to the equation as \eqref{equation}.
\usepackage{caption}  % This package allows captioning outside of a float
\usepackage[export]{adjustbox}


\usetikzlibrary{patterns}

\usetikzlibrary{patterns.meta}

\usepackage[para]{footmisc} % Example of making footnotes run together in a paragraph
\definecolor{darkgreen}{rgb}{0.0, 0.5, 0.0}  % Darker green

\usepackage{datetime}

\usepackage{xcolor}


\usepackage{etoolbox}

\makeatletter
\def\tagform@#1{\maketag@@@{{Eq.~#1}}} % Adds "Eq." and makes it bold
\makeatother

\begin{document}
	
	\vspace*{\fill}
	\begin{center}
		\textbf{\Huge Laboratory Report}\\[10pt]
		\LARGE \textbf{Pressure \& Refrigeration}
	\end{center}
	\vspace*{\fill}
	
	\Large    
	\begin{tabular}{@{}l l l@{}}
		\textbf{Submitted by:} & Sakariye Abiikar & K2371673 \\
		& Alireza Alishahi & K2333243 \\
		& Naim Alrifai & K2459662 \\
		& Munachi J Atuegbu & K2463699 \\
		& Ehsan Haque & K2453799 \\
		& Abdul Mueed & K2454880 \\   
		& Abdelrahman Shehata & K2426523 \\   
		& Varley, Freddie & K2311322
	\end{tabular}
	
	\vspace*{\fill}
	
	\begin{tabular}{@{}l l@{}}
		\textbf{Key Dates:} & Date of practical: Wednesday 19$^{\text{th}}$ March, 2025 \\
		& Deadline: Tuesday 3$^{\text{rd}}$ April, 2025 \\
		& Last Updated: \today\, \currenttime\\
	\end{tabular}
	\vspace*{\fill}
	
	\large
	\newpage\noindent	\vspace*{-1em}
	
	\begin{center}
		\LARGE \textbf{Contribution Table}\\[3em]
	\end{center}	
	
	
	\begin{tblr}{
			colspec={Q[4cm]Q[4cm]Q[4cm]Q[3cm]},
			hlines,vlines,
			cells={valign=m,halign=c},
			rows={ht=4\baselineskip},
			row{1}={ht=1.5\baselineskip,font=\bfseries},
		}
		Student & Course & Contribution & Picture \\ 
		Sakariye Abiikar & Mechanical Engineering & ~ & \includegraphics[width=2cm,valign=c]{images/image(7).jpeg} \\ 
		Alireza Alishahi & Mechanical Engineering  & ~ & \includegraphics[width=2cm,valign=c]{images//profile.jpg} \\ 
		Naim Alrifai & Mechanical Engineering & ~ & \includegraphics[width=2cm,valign=c]{images/profile.jpg} \\ 
		Munachi J Atuegbu & Mechanical Engineering & ~  & \includegraphics[width=2cm,valign=c]{images//profile.jpg} \\ 
		Ehsan Haque & Mechanical Engineering & ~ & \includegraphics[width=2cm,valign=c]{images//profile.jpg} \\
		Abdul Mueed & Mechanical Engineering  & ~ & \includegraphics[width=2cm,valign=c]{images/profile.jpg} \\ 
		Abdelrahman Shehata & Mechanical Engineering  & ~ & \includegraphics[width=2cm,valign=c]{images/profile.jpg}\\
		Varley, Freddie & Mechanical Engineering & &  \includegraphics[width=2cm,valign=c]{images/profile.jpg} 
	\end{tblr}
		\vspace*{\fill}
	
	\normalsize
	\newpage\newgeometry{top=0.2in,bottom=0.6in,left=0.8in,right=0.8in}
	\noindent\vspace{0em}
	\begin{center}
		\LARGE \textbf{Table of Contents}\\[-7em]
	\end{center}
	{
		\hypersetup{linkcolor=black}
		\tableofcontents
	}    
	
	
	\large\newpage\restoregeometry\vspace*{-20pt}
	
	\noindent
	
\section{Abstract}
In this experiment, we calibrated and compared the performance of multiple pressure-measuring devices, including two Bourdon gauges, a Budenberg pressure gauge and a Hg glass manometer, in tandem with a reference pressure calibrator (DPI-603 Portable Pressure Calibrator), which served as the baseline for pressure measurements and as the source of the applied pressure. The devices were connected to the DPI-603 Portable Pressure Calibrator, enabling us to apply both positive and negative pressures in increments of approximately $\pm$5 kPa. Through this process, we were able to get an exhaustive dataset that demonstrated notable differences in the pressure-measuring devices' performance. \textcolor{red!50!white}{[Placeholder breif results findings]}. These results highlighted the importance of selecting appropriate devices based on precision requirements and operating conditions, as well as the potential impact of human error in reading analog instruments like the Bourdon gauges and Hg manometer. 

	
	\newpage\vspace*{-20pt}

	\section{Introduction}
	In engineering and scientific applications, pressure measurement is essential, playing a critical role in fields such as fluid dynamics, meteorology, and industrial control. Over time, pressure-measuring instruments have evolved, from early liquid column manometers to modern mechanical and digital gauges, each designed to provide accurate measurements under varying conditions.\\[1em]
	A key distinction in pressure measurement is between \textbf{absolute} and \textbf{gauge} pressure \footnote{\textbf{Differential} pressure i decided to redact in the context of our lab though it would be among this list}.\\ 
	Absolute pressure is measured relative to a vacuum, while gauge pressure is measured relative to atmospheric pressure. This distinction influences the design and function of pressure-measuring devices.\\[1em]
	Historically, the invention of the Bourdon gauge by Eugène Bourdon in 1849 marked a significant advancement in pressure measurement. It provided a robust and reliable means of monitoring pressure in industrial settings, where durability and consistency were paramount. On the other hand, liquid column manometers, particularly those using mercury, have been essential in laboratory settings due to their precision in measuring small pressure differences.\\[1em]
	Despite the advantages of different pressure-measuring devices, each has limitations, such as calibration errors and environmental influences.\\[1em]
	Previous studies have highlighted various challenges and findings related to pressure measurement devices. For example, a study by Hodgkinson et al. (2020) found that the accuracy of home blood pressure monitors varied significantly, with validated monitors showing a higher pass rate in static pressure tests compared to unvalidated ones. This emphasizes the importance of validation and calibration in ensuring the accuracy of pressure-measuring devices.\\[1em]
	In this experiment, we aimed to evaluate the performance and accuracy of various pressure-measuring devices under controlled conditions, comparing their responses to varying pressure levels.

	
	\newpage\vspace*{-20pt}
	\section{Method \& Experimental Procedures}\label{Method_Experimental_Procedures}
	%We are first given a table to record our findings as shown:\footnote{We started of in the lab by doing the refrigerator but this is redacted as its not a part of this lab report, following this we did the pressure lab experiment.}

	\begin{figure}[H] 
			\centering 
		    \begin{tikzpicture}[auto,
			block/.style ={rectangle, draw=blue, thick, fill=blue!20, text width=1em,align=center, rounded corners, minimum height=2em},
			block1/.style ={rectangle, draw=blue, thick, fill=blue!20, text width=5em,align=center, rounded corners, minimum height=2em},
			line/.style ={draw, thick, -latex',shorten >=2pt},
			cloud/.style ={draw=red, thick, ellipse,fill=red!20,
				minimum height=1em}]
			\node[anchor=center] (image) at (-10,0) {\includegraphics[width=0.9\textwidth]{extracted_images/image_7_1.png}};
			\node[block] (A) at (-9.85,-4.5) {1};
			\node[block] (B) at (-7.15,-4.5) {2};
			\node[block1] (Gauge) at (-8.5,-3) {Bourdon Gauge};
			\draw[line] (Gauge.south) -- ++(0,-0.25) -| (A.north);
			\draw[line] (Gauge.south) -- ++(0,-0.25) -| (B.north);			
			\node[block1] (Gauge) at (-15.5,-4.2) {Pressure calibrator};
			\node[block1] (Gauge) at (-7.5,-1.1) {Pressure Gauge};
			\node[block1] (Gauge) at (-3.5,7.6) {Mercury Glass Manometer};
		\end{tikzpicture}
		\caption{Pressure Measurement Bench} 
		\label{fig:pressure_measurement_bench} 
	\end{figure}
	
	\begin{figure}[H] 
			\centering 
			\includegraphics[width=1\textwidth,cfbox=gray!15 1pt]{images/tableland-1_page-0001.jpg} 
			\caption{Form for recording pressure measurements} 
			\label{fig:pressurestable} 
	\end{figure}
	\underline{The experiment \textbf{summary} is as follows:}\\[1em]
	The instructor provided us with a comprehensive demonstration on how to operate the pressure calibrator, guiding us through the process of setting up the equipment and interpreting the corresponding gauge readings.\\[1em]
	Prior to starting the experiment, we were given essential safety instructions, including precautions to prevent compromising our data and ensuring the safety of everyone involved.\\[1em]
	As the experiment progressed, we worked as a team to carefully apply the required pressures, analyzing the gauge readings and reaching a consensus on the correct values, which were then recorded in the designated pressure measurement table (Figure \ref{fig:pressurestable}).\\[1em]
	A detailed breakdown of the \textbf{exact steps} we followed is as follows:
\newpage
\newgeometry{left=0.8in,right=0.8in,top=1.3in,bottom=0.6in}
\begin{minipage}{0.5\textwidth}	
	\subsection{Operating Procedure}
	\begin{enumerate}[left=0in,itemsep=2mm]
	    \item The instructor inspected the test rig’s pneumatic connections to ensure they were secure.  
		\item The instrument’s vent valve was opened as part of the setup process.  
		\item Given the choice between vacuum (\textsf{\textcolor{blue}{-}}) and excess (\textsf{\textcolor{red}{+}}) pressure, we initially set the selector on the front of the DPI-603 ($\pm$VE in Figure \ref{fig:DPI-603}) to positive pressure. This setting allowed us to apply the necessary excess pressure for the procedure.  
		\item The unit was powered on by pressing the power button.  
		\item Using the pressure units button, we cycled through the available options (in Hg, bar, etc.) and selected kPa.  
	\end{enumerate}
\end{minipage}\hfill
\begin{minipage}{0.45\textwidth}
	\begin{figure}[H] 
		\centering 		
		\hspace*{-2em}
		\begin{tikzpicture}[scale=0.45,transform shape]
			\node[anchor=center] (image) at (0,0) {\includegraphics[width=1\textwidth]{extracted_images/image_8_1.png}};
     \draw[{Latex[length=3mm,width=3mm]}-,line width=0.5mm] (3.2, -2.2) -- (5, -3) node[above=1mm,right] {\Huge Vent Valve}; 
		     
		    \draw[{Latex[length=3mm,width=3mm]}-,line width=0.5mm] (-2, 3.4) -- (-5, 3) node[left=14mm,below=-10mm] {\Huge \wm{1}{Pressure\\ Units}}; 
		
		\draw[{Latex[length=3mm,width=3mm]}-,line width=0.5mm] (-0.15, -5) -- (-0.15, -8) node[below=1mm] {\Huge $\pm \text{VE}$}; 
		
		\draw[{Latex[length=3mm,width=3mm]}-,line width=0.5mm] (1.35, -6.3) -- (3, -8) node[below=6.5mm,right] {\Huge\text{Hand Pump}}; 
		
		\draw[{Latex[length=3mm,width=3mm]}-,line width=0.5mm] (1.35-3, -6.3) -- (-3, -8) node[below=5.9mm,left] {\Huge\text{Volume Control}}; 
		
		\draw[{Latex[length=3mm,width=3mm]}-,line width=0.5mm] (-1.5, 5) -- (-3.5, 7.4) node[above=5.9mm,left=-1cm] {\Huge\text{Main Display}}; 
		
		\draw[{Latex[length=3mm,width=3mm]}-,line width=0.5mm,draw=black] (2.5, 5) -- (4.3, 7.4) node[above=5.9mm,right=-1cm] {\Huge\text{Power}}; 
		
		\end{tikzpicture}
		\caption{DPI-603 Portable Pressure Calibrator} 
		\label{fig:DPI-603} 
	\end{figure}
\end{minipage}\\
\begin{enumerate}
\item[6.] The vent valve was closed and used to zero the instrument. It was concluded that this procedure should be carried out by the instructor due to the vent valve’s sensitivity.  
\end{enumerate}
\noindent
Here Steps 1–6 primarily cover the setup phase and were mainly carried out by the instructor. It was now our role to conduct the rest of the experiment, which proceeded as follows:

\begin{enumerate}
\item[7.] We used the hand pump to pressurize the system to the required value ($\approx\pm 5$ kPa). To achieve precise control, we vented air using the vent valve and adjusted the pressure by pumping air as needed.
\item[8.] Once the required incremtal $\pm 5$ kPa was observed on the pressure calibrator, we then observe and recorded the readings seen on the following gauges:
\end{enumerate}

	\begin{minipage}{0.45\textwidth}\centering
		\begin{center}
			\begin{minipage}{0.45\textwidth}\centering
				\includegraphics[width=1.2\textwidth]{images/Image(1).jpg}\\[0.5em]
				\rotatebox[origin=c]{0}{\includegraphics[width=1.2\textwidth]{images/Image(6).jpg}}
			\end{minipage}\hspace{0.5em}
			\begin{minipage}{0.4\textwidth}\centering
				\includegraphics[width=0.55\textwidth]{images/Image(2).jpg}
			\end{minipage}
			\captionof{figure}{Gauges}\small See Figure \ref{fig:pressure_measurement_bench} for references as to see what gauge is what
		\end{center}
	\end{minipage}
	\begin{minipage}{0.51\textwidth}\vspace{-2em}\raggedright
		\begin{enumerate}[left=0in]
			\item[9.]  By reaching a consensus as a team over what is being read on the instruments for this related pressure value on the calibrator, we recorded them in the designated pressure measurement table (Figure \ref{fig:pressurestable}). 
			\item[10.] This procedure (Steps 7-9) is repeated 10 times, Starting the \textbf{initial reading of 0kPa} and ending at 50kPa. 
			\item[11.] Once done, we reverse the process (Steps 6-10), in which we proceed to do the same for vacuum.
		\end{enumerate}\noindent
		This concludes all that was done in the laboratory\\ for us to conclude results based on the data gathered in the tables.
	\end{minipage}
	
	\newpage\restoregeometry
	\section{Theory}
	
	\subsection{Pressure}
	
	When we talk about pressure, the first thing that comes to mind is its physical definition. It refers to the effect or various types of deflection when a force is applied to a surface.
	
	\begin{figure}[h]
		\centering
		\begin{minipage}{0.45\textwidth}
			\centering
			\begin{equation}
				P = \frac{F}{A}
				\label{eq:pressure}
			\end{equation}
			\wm{0.7}{
			\begin{itemize}[itemsep=-1mm]
				\item \( P \) = Pressure (Pa, Pascal)
				\item \( F \) = Force applied (N, Newton)
				\item \( A \) = Surface area (m\(^2\))
			\end{itemize}
			}\\[1em]\raggedright
			"Pressure is the force per unit area exerted normal to the surface."
		\end{minipage}
		\hfill
		\begin{minipage}{0.45\textwidth}
			\centering
			\includegraphics[width=\textwidth]{images/Pressure.jpg}
			\caption{Illustration of pressure application}
			\label{fig:pressure}
		\end{minipage}
	\end{figure}
	\noindent
	The type of deflection caused by pressure depends on its magnitude and the duration of application. Pressure has numerous examples and applications, such as vehicle tires and press machines.\\[1em]
	The most effective way to estimate applied pressure is by measuring its effects, for example, by observing changes in the dimensions of an elastic object or the height of a liquid column, as shown in the following image.
	\begin{figure}[H]
		\centering
		\includegraphics[width=0.5\textwidth]{images/Piezometer.jpg}
		\caption{Pressure measurement using a liquid column}
		\label{fig:piezometer}
	\end{figure}
	\noindent
	Pressure is categorized into two main groups: \textbf{gauge pressure} and \textbf{absolute pressure}, explained in the following diagram.
	\subsection{Absolute Pressure}
	Absolute pressure, or absolute zero pressure, is the lowest possible pressure measurable. Consequently, all measured pressures are positive in comparison to this reference point. Achieving absolute zero pressure is practically impossible unless calculated or represented through an extremely accurate curve.
	\subsection{Gauge Pressure}
	Gauge pressure, also known as relative pressure, is measured relative to local atmospheric pressure. Since we live under constant atmospheric pressure, it is often convenient to measure the difference between actual pressure and atmospheric pressure, which is referred to as gauge pressure. This measurement is commonly used in industrial applications.\\[-0.2em]
	\hrule\vspace{0.8em}
	It is important to note that:\\[-5pt]
	\begin{itemize}
		\item Any pressure between local atmospheric pressure and absolute zero pressure is called \textbf{vacuum pressure}.
		\item Any pressure higher than local atmospheric pressure is considered \textbf{positive pressure}.
	\end{itemize}
	\vspace{0.7em}\noindent
	The relationship between absolute and gauge pressure is given by:\\[0.5em]
	\begin{equation}
		P_{\text{absolute}} = P_{\text{atmosphere}} + P_{\text{gauge}}
		\label{eq:absolute}
	\end{equation}\\
	\vspace{0.5em}
	The following diagram illustrates the definitions of pressure more clearly.	
	\begin{figure}[H]
		\centering
		\includegraphics[width=0.8\textwidth]{images/Pressure Diagram.jpg}
		\caption{Pressure categories and their relationships}
		\label{fig:pressure_diagram}
	\end{figure}
	
	\subsection{Other Types of Pressure in Flow Measurement}
	Additionally, different interpretations of pressure are applicable in flow measurement. The most common types of pressure in this context are \textbf{total}, \textbf{static}, and \textbf{dynamic} pressures.
	
	\subsubsection{Total Pressure}
	Total pressure is the force per unit area felt when a flowing fluid is brought to rest. It is typically measured using a Pitot tube, as shown in the figure below. Total pressure is the sum of static pressure and dynamic pressure:\\[0.5em]
	\begin{equation}
		P_{\text{total}} = P_{\text{static}} + P_{\text{dynamic}}
		\label{eq:total}
	\end{equation}\\
	Total pressure is also referred to as stagnation pressure.
	\subsubsection{Static Pressure}
	Static pressure is the force exerted on a fluid particle from all directions when the fluid is at rest or when measured in a moving reference frame. It is commonly measured using gauges or transmitters attached to the side of a pipe or tank wall. In most discussions, the term "pressure" generally refers to static pressure.
	\subsubsection{Dynamic Pressure}
	Dynamic pressure represents the kinetic energy of a flowing fluid and is the difference between total and static pressure. It is given by:
	\begin{equation}
		P_{\text{dynamic}} = \frac{\rho v^2}{2g}
		\label{eq:dynamic}
	\end{equation}	
	\begin{figure}[H]
		\centering
		\includegraphics[width=0.75\textwidth]{images/Flow Rate.jpg}
		\caption{Measurement of dynamic pressure in fluid flow}
		\label{fig:flow_rate}
	\end{figure}
	
	\subsection*{Pressure Units}
	The SI unit for pressure is the Pascal (Pa), which is defined as one Newton per square meter (N/m\(^2\)). In other unit systems, pressure is often measured in pounds per square inch (psi). However, one of the most widely used units is the bar or millibar (mbar). Some common pressure units and their equivalents are listed below:
	
	\begin{itemize}
		\item \(1\) bar = \(10^5\) Pa
		\item \(1\) atm = \(101325\) Pa
		\item \(1\) psi = \(6894.76\) Pa
		\item \(1\) mmHg = \(133.322\) Pa
	\end{itemize}
	
	
	
	
	\newpage	
	\section{Data, Calculations and Results}
	Thus, as explained in the Method \& Experimental Procedures Section \ref{Method_Experimental_Procedures}, we have gathered a simple dataset for the corresponding pressure gauges. Our goal now is to make calculations, illustrate trends, etc., and simply produce another dataset from which we can draw fair conclusions from. Here, I first describe the baseline data that was collected and recorded on the table that was made available for use in the lab (Figure \ref{figure:s}).
	\subsection{Pressure Calibrator Readings}
	\vspace{-1em}
	\begin{table}[H]
		\centering
		\begin{minipage}[t]{0.17\textwidth}
			\centering
			\textbf{\textsf{kPa}}\\[8pt]
			
			\begin{tblr}{
					colspec={X[1cm]X[1cm]},
					hlines,vlines,
					cells={valign=m,halign=c},
					rows={ht=1\baselineskip},
					row{1}={ht=1\baselineskip,font=\bfseries},
				}
				\Large\textsf{\textcolor{red}{+}}&\wm{0.2}{\vspace{0.1cm}\Large\textsf{\textcolor{blue}{-}}}\\\hline
				0.0  & 0.0  \\
				5.7  & -5.6  \\
				10.4 & -12.1 \\
				16.0 & -18.0 \\
				21.1 & -21.8 \\
				27.7 & -25.4 \\
				34.2 & -29.3 \\
				40.0 & -33.6 \\
				46.1 & -37.6 \\
				52.2 & -41.7 \\
			\end{tblr}
			\caption{Pressure Calibrator}
		\end{minipage}
		\hfill
		\begin{minipage}[t]{0.17\textwidth}
			\centering
			\textbf{\textsf{psi}}\\[8pt]
			
			\begin{tblr}{
					colspec={X[1cm]X[1cm]},
					hlines,vlines,
					cells={valign=m,halign=c},
					rows={ht=1\baselineskip},
					row{1}={ht=1\baselineskip,font=\bfseries},
				}
				\Large\textsf{\textcolor{red}{+}}&\wm{0.2}{\vspace{0.1cm}\Large\textsf{\textcolor{blue}{-}}}\\\hline
				1.0  & 1.2  \\
				2.0  & 0.4  \\
				2.6  & -0.5  \\
				3.4  & -2.0  \\
				4.1  & -2.8  \\
				5.0  & -4.0  \\
				6.0  & -6.0  \\
				6.8  & -7.1  \\
				7.6  & -8.3  \\
				8.5  & -9.5  \\
			\end{tblr}
			\caption{Bourdon Gauge 1}
		\end{minipage}
		\hfill
		\begin{minipage}[t]{0.17\textwidth}
			\centering
			\textbf{\textsf{kN/m$\bm{^2}$}}\\[8pt]
			
			\begin{tblr}{
					colspec={X[1cm]X[1cm]},
					hlines,vlines,
					cells={valign=m,halign=c},
					rows={ht=1\baselineskip},
					row{1}={ht=1\baselineskip,font=\bfseries},
				}
				\Large\textsf{\textcolor{red}{+}}&\wm{0.2}{\vspace{0.1cm}\Large\textsf{\textcolor{blue}{-}}}\\\hline
				1.0  & 2.5  \\
				8.0  & -1.0  \\
				14.0 & -9.0  \\
				20.0 & -15.0 \\
				25.0 & -20.0 \\
				30.0 & -23.0 \\
				39.0 & -27.0 \\
				45.0 & -32.0 \\
				50.0 & -36.0 \\
				57.0 & -40.0 \\
			\end{tblr}
			\caption{Bourdon Gauge 2}
		\end{minipage}
		\hfill
		\begin{minipage}[t]{0.17\textwidth}
			\centering
			\textbf{\textsf{bar}}\\[8pt]
			
			\begin{tblr}{
					colspec={X[1cm]X[1cm]},
					hlines,vlines,
					cells={valign=m,halign=c},
					rows={ht=1\baselineskip},
					row{1}={ht=1\baselineskip,font=\bfseries},
				}
				\Large\textsf{\textcolor{red}{+}}&\wm{0.2}{\vspace{0.1cm}\Large\textsf{\textcolor{blue}{-}}}\\\hline
				-0.05 & -0.05  \\
				0.00  & -0.10  \\
				0.04  & -0.16  \\
				0.10  & -0.24  \\
				0.15  & -0.27  \\
				0.22  & -0.30  \\
				0.29  & -0.35  \\
				0.35  & -0.40  \\
				0.40  & -0.44  \\
				0.47  & -0.49  \\
			\end{tblr}
			\caption{Budenberg Pressure Gauge}
		\end{minipage}
		\hfill
		\begin{minipage}[t]{0.17\textwidth}
			\centering
			\textbf{\textsf{cm Hg}}\\[8pt]
			
			\begin{tblr}{
					colspec={X[1cm]X[1cm]},
					hlines,vlines,
					cells={valign=m,halign=c},
					rows={ht=1\baselineskip},
					row{1}={ht=1\baselineskip,font=\bfseries},
				}
				\Large\textsf{\textcolor{red}{+}}&\wm{0.2}{\vspace{0.1cm}\Large\textsf{\textcolor{blue}{-}}}\\\hline
				0.4  & 0.4  \\
				3.5  & -0.7  \\
				5.3  & -3.7  \\
				7.4  & -5.4  \\
				9.4  & -6.8  \\
				11.6 & -8.2  \\
				14.2 & -9.6  \\
				16.4 & -11.3 \\
				18.7 & -12.8 \\
				21.0 & -14.4 \\
			\end{tblr}
			\caption{Hg Glass Manometer}
		\end{minipage}
	\end{table}
	
	\newpage
	\section{Discussion of Results}
	
	\newpage\vspace*{-30pt}
	\section{Conclusions}
	\newpage\vspace*{-30pt}
	
	\section{Recommendations}  		
	\newpage\vspace*{-30pt}
	
	
	
	\section{References}	
	\begin{enumerate}
		\item Hodgkinson, J.A. et al. (2020). \textit{Accuracy of blood-pressure monitors owned by patients with hypertension (ACCU-RATE study): a cross-sectional, observational study in central England}. British Journal of General Practice, [online] 70(697), pp.e548–e554. Available at: \url{https://bjgp.org/content/70/697/e548.}
	\end{enumerate}
	‌	
	\newpage\vspace*{-30pt}
		
	
	\section{Appendix}
	
\end{document}
