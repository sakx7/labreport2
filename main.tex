\documentclass{article}
\usepackage{graphicx} 
\usepackage{float}
\usepackage{booktabs}
\usepackage{array}
\usepackage{arydshln}
\usepackage{siunitx}
\usepackage{hyperref}
\usepackage{cancel}
\usepackage{changepage}
\usepackage{placeins}
\usepackage{enumitem}
\usepackage{siunitx}
%\usepackage{showframe}
\usepackage{times}

\usepackage{tabularx}
\usepackage{amsmath, amssymb, amscd, MnSymbol, mathrsfs}
\usepackage{cellspace}
\usepackage{tikz}
\usetikzlibrary{calc, patterns, angles, quotes, decorations.markings, decorations.pathmorphing, hobby}
\usepackage{xfrac}

\usepackage{chemfig}
\usepackage{caption}
\usepackage{tcolorbox}
\usepackage{bm}
\usepackage{pdfpages}
\usepackage{empheq}
\usepackage{pgfplots}
\pgfplotsset{compat=1.18}
\usepackage[oldvoltagedirection]{circuitikz}
\usepackage{microtype}
\usepackage{tikz-3dplot}
\usepackage{textcomp}
% Custom commands
\newcommand{\vect}[1]{\boldsymbol{\mathbf{#1}}}
\newcolumntype{C}{>{\centering\arraybackslash}X}
\newcolumntype{M}[1]{>{\centering\arraybackslash}m{#1}}

\usetikzlibrary{external}
\tikzexternalize[prefix=figures/]

\newcommand\myfrac[2]{\sfrac{#1\mkern-1.2mu}{#2}}
\usepackage{xcolor}

% Define custom colors
\definecolor{darkblue}{rgb}{0.1,0.1,0.5} % A dark blue shade
\definecolor{formalshade}{rgb}{0.95,0.95,1} % A light blue shade for the background

% For the adjustwidth environment
\PassOptionsToPackage{strict}{changepage}
\usepackage{changepage}

% For formal definitions
\usepackage{framed}

\newcommand{\formalsource}{} % Initialize an empty macro to store the source text

\newenvironment{formal}[1][]{% Start of the environment
	\renewcommand{\formalsource}{#1}% Store the optional argument
	\def\FrameCommand{%
		\hspace{1pt}%
		{\color{gray}\vrule width 2pt}%
		{\color{white}\vrule width 4pt}%
		\colorbox{white}%
	}%
	\MakeFramed{\advance\hsize-\width\FrameRestore}%
	\noindent\hspace{-4.55pt}% Disable indenting the first paragraph
	\begin{adjustwidth}{}{7pt}%
		\vspace{2pt}%
	}%
	{%
		\vspace{4pt}%
		\ifx\formalsource\empty % Check if the source is empty
		\else
		\hfill{\footnotesize{\formalsource}}% Align source to the bottom-right
		\fi
	\end{adjustwidth}\endMakeFramed%
}


% Custom itemize list with images for positive and negative items
\newlist{gitemize}{itemize}{1} % Just one level for the list
\setlist[gitemize,1]{
	leftmargin=2.8em, % Adjust the margin for the list
	labelsep=1em % Control the space between the label and the list item
}

% Define checkmark and cross symbols for positive and negative items
\newcommand{\checkitem}{\raisebox{-0.25\height}{\includegraphics[width=0.4cm]{checkmark.png}}}
\newcommand{\crossitem}{\raisebox{-0.25\height}{\includegraphics[width=0.4cm]{cross.png}}}


\usepackage[left=0.8in,right=0.8in,top=0.5in,bottom=0.69in,includeheadfoot,letterpaper]{geometry}
\usepackage{fancyhdr}
\usepackage{graphicx}
\usepackage{tabularray}
\usepackage{varwidth} 


\newcommand{\wm}[1]{%
	\begin{minipage}{1\textwidth}
		#1
	\end{minipage}%
}

\pagestyle{fancy}
\fancyhf{}


\renewcommand{\headrulewidth}{0.4pt}
\renewcommand{\footrulewidth}{0.4pt}
\usepackage{mathtools}
\fancyhead[L]{\includegraphics[height=1.2cm]{images/Kingston_University_London_logo_200-tablet.png}}
\fancyhead[R]{EG4024 – ME – Fluid Mechanics and Thermodynamics}
\fancyfoot[C]{Department of Mechanical Engineering}
\fancyfoot[R]{\thepage}

\usepackage{scalerel}

\setlength{\headheight}{30pt}
\setlength{\footskip}{20pt}

%\fancypagestyle{plain}{
	%    \fancyhf{} % Clear all header and footer fields for plain style
	%    \fancyhead[L]{\includegraphics[height=1.2cm]{images/Kingston_University_London_logo_200-tablet.png}} % Left header
	%    \fancyhead[R]{EG4019 - ME - Engineering Mechanics and Materials} % Right header
	%    \fancyfoot[C]{Department of Mechanical Engineering} % Center footer
	%    \fancyfoot[R]{\thepage} % Right footer (page number)
	%}


\usepackage[export]{adjustbox}
\usepackage{tocloft}
\renewcommand{\cfttoctitlefont}{}
\renewcommand{\contentsname}{}
\renewcommand{\cftsecleader}{\cftdotfill{\cftdotsep}}

\setlength{\cftbeforesecskip}{0.5em}


\usepackage{xcolor}      % For color options
\usepackage{hyperref}    % For hyperlinks
\usepackage{xurl}        % For better URL handling
\hypersetup{
	colorlinks=true,
	linkcolor=blue!50!black,
	urlcolor=blue,       % Color for URLs
}



%Refer to the equation as \eqref{equation}.
\usepackage{caption}  % This package allows captioning outside of a float
\usepackage[export]{adjustbox}


\usetikzlibrary{patterns}

\usetikzlibrary{patterns.meta}

\pgfdeclarepattern{
	name=hatch,
	parameters={\hatchsize,\hatchangle,\hatchlinewidth},
	bottom left={\pgfpoint{-.1pt}{-.1pt}},
	top right={\pgfpoint{\hatchsize+.1pt}{\hatchsize+.1pt}},
	tile size={\pgfpoint{\hatchsize}{\hatchsize}},
	tile transformation={\pgftransformrotate{\hatchangle}},
	code={
		\pgfsetlinewidth{\hatchlinewidth}
		\pgfpathmoveto{\pgfpoint{-.1pt}{-.1pt}}
		\pgfpathlineto{\pgfpoint{\hatchsize+.1pt}{\hatchsize+.1pt}}
		\pgfpathmoveto{\pgfpoint{-.1pt}{\hatchsize+.1pt}}
		\pgfpathlineto{\pgfpoint{\hatchsize+.1pt}{-.1pt}}
		\pgfusepath{stroke}
	}
}

\tikzset{
	hatch size/.store in=\hatchsize,
	hatch angle/.store in=\hatchangle,
	hatch line width/.store in=\hatchlinewidth,
	hatch size=5pt,           % Smaller hatch size for fewer lines
	hatch angle=45pt,         % More angle to spread lines
	hatch line width=.5pt,    % Thin lines
}

\usepackage[para]{footmisc} % Example of making footnotes run together in a paragraph
\definecolor{darkgreen}{rgb}{0.0, 0.5, 0.0}  % Darker green




\begin{document}
	
	\vspace*{\fill}
	\begin{center}
		\textbf{\Huge Laboratory Report}\\[10pt]
		\LARGE \textbf{Pressure \& Refrigeration}
	\end{center}
	\vspace*{\fill}
	
	\Large    
	\begin{tabular}{@{}l l l@{}}
		\textbf{Submitted by:} & Sakariye Abiikar (Group Leader)\phantom{ssssss} & K2371673 \\
		& $\zeta$~ & K~ \\
		& $\zeta$~ &  K~ \\
		& $\zeta$~ & K~ \\
		& $\zeta$~ & K~ \\
		& $\zeta$~ & K~ \\   
	\end{tabular}
	
	\vspace*{\fill}
	
	\begin{tabular}{@{}l l@{}}
		\textbf{Key Dates:} & Date of practical: Varies\\
		& Deadline: 31/12/2024 \\
		& Date of submission: 31/12/2024\\
	\end{tabular}
	\vspace*{\fill}
	
	\large
	\newpage\noindent\vspace{2em}
	\begin{center}
		\LARGE \textbf{Contribution Table}\\[3em]
	\end{center}
	
	
	
	\begin{tblr}{
			colspec={Q[4cm]Q[4cm]Q[4cm]Q[3cm]},
			hlines,vlines,
			cells={valign=m,halign=c},
			rows={ht=4\baselineskip},
			row{1}={ht=1.5\baselineskip,font=\bfseries},
		}
		Student & Course & Contribution & Picture \\ 
		Sakariye Abiikar & Mechanical Engineering & ~ & \includegraphics[width=2cm,valign=c]{images/profile.jpg} \\ 
		~ & Mechanical Engineering  & ~ & \includegraphics[width=2cm,valign=c]{images//profile.jpg} \\ 
		~ & Mechanical Engineering & ~ & \includegraphics[width=2cm,valign=c]{images/profile.jpg} \\ 
		~ & Mechanical Engineering & ~  & \includegraphics[width=2cm,valign=c]{images//profile.jpg} \\ 
		~ & Mechanical Engineering & ~ & \includegraphics[width=2cm,valign=c]{images//profile.jpg} \\
		~ & Mechanical Engineering  & ~ & \includegraphics[width=2cm,valign=c]{images/profile.jpg} \\ 
	\end{tblr}
	
	\normalsize
	\newpage\newgeometry{top=0.2in,bottom=0.6in,left=0.8in,right=0.8in}
	\noindent\vspace{0em}
	\begin{center}
		\LARGE \textbf{Table of Contents}\\[-7em]
	\end{center}
	{
		\hypersetup{linkcolor=black}
		\tableofcontents
	}    
	
	
	\large\newpage\restoregeometry\vspace*{-20pt}
	
	\section{Abstract}
	\vspace*{1em}
	
	
	\newpage\vspace*{-20pt}
	\section{Introduction}
	
	
	\newpage\vspace*{-20pt}
	\section{Method}
	
	\newpage
	
	\section{Experimental Procedures}
	
	\newpage
	\section{Conducting the Experiment}
	
	\newpage
	\section{Theory}
	
	\newpage	
	\section{Data, Calculations and Results}
	
	\newpage
	\section{Discussion of Results}
	
	\newpage\vspace*{-30pt}
	\section{Conclusions}
	\newpage\vspace*{-30pt}
	
	\section{Recommendations}  		
	\newpage\vspace*{-30pt}
	
	
	\section{References}		
	\newpage\vspace*{-30pt}
	
	
	\section{Appendix}
	
\end{document}
